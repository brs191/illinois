% Options for packages loaded elsewhere
\PassOptionsToPackage{unicode}{hyperref}
\PassOptionsToPackage{hyphens}{url}
\PassOptionsToPackage{dvipsnames,svgnames*,x11names*}{xcolor}
%
\documentclass[
]{article}
\usepackage{lmodern}
\usepackage{amssymb,amsmath}
\usepackage{ifxetex,ifluatex}
\ifnum 0\ifxetex 1\fi\ifluatex 1\fi=0 % if pdftex
  \usepackage[T1]{fontenc}
  \usepackage[utf8]{inputenc}
  \usepackage{textcomp} % provide euro and other symbols
\else % if luatex or xetex
  \usepackage{unicode-math}
  \defaultfontfeatures{Scale=MatchLowercase}
  \defaultfontfeatures[\rmfamily]{Ligatures=TeX,Scale=1}
\fi
% Use upquote if available, for straight quotes in verbatim environments
\IfFileExists{upquote.sty}{\usepackage{upquote}}{}
\IfFileExists{microtype.sty}{% use microtype if available
  \usepackage[]{microtype}
  \UseMicrotypeSet[protrusion]{basicmath} % disable protrusion for tt fonts
}{}
\makeatletter
\@ifundefined{KOMAClassName}{% if non-KOMA class
  \IfFileExists{parskip.sty}{%
    \usepackage{parskip}
  }{% else
    \setlength{\parindent}{0pt}
    \setlength{\parskip}{6pt plus 2pt minus 1pt}}
}{% if KOMA class
  \KOMAoptions{parskip=half}}
\makeatother
\usepackage{xcolor}
\IfFileExists{xurl.sty}{\usepackage{xurl}}{} % add URL line breaks if available
\IfFileExists{bookmark.sty}{\usepackage{bookmark}}{\usepackage{hyperref}}
\hypersetup{
  pdftitle={Week 4 - Homework},
  pdfauthor={STAT 420, Summer 2020, D. Unger},
  colorlinks=true,
  linkcolor=Maroon,
  filecolor=Maroon,
  citecolor=Blue,
  urlcolor=cyan,
  pdfcreator={LaTeX via pandoc}}
\urlstyle{same} % disable monospaced font for URLs
\usepackage[margin=1in]{geometry}
\usepackage{color}
\usepackage{fancyvrb}
\newcommand{\VerbBar}{|}
\newcommand{\VERB}{\Verb[commandchars=\\\{\}]}
\DefineVerbatimEnvironment{Highlighting}{Verbatim}{commandchars=\\\{\}}
% Add ',fontsize=\small' for more characters per line
\usepackage{framed}
\definecolor{shadecolor}{RGB}{248,248,248}
\newenvironment{Shaded}{\begin{snugshade}}{\end{snugshade}}
\newcommand{\AlertTok}[1]{\textcolor[rgb]{0.94,0.16,0.16}{#1}}
\newcommand{\AnnotationTok}[1]{\textcolor[rgb]{0.56,0.35,0.01}{\textbf{\textit{#1}}}}
\newcommand{\AttributeTok}[1]{\textcolor[rgb]{0.77,0.63,0.00}{#1}}
\newcommand{\BaseNTok}[1]{\textcolor[rgb]{0.00,0.00,0.81}{#1}}
\newcommand{\BuiltInTok}[1]{#1}
\newcommand{\CharTok}[1]{\textcolor[rgb]{0.31,0.60,0.02}{#1}}
\newcommand{\CommentTok}[1]{\textcolor[rgb]{0.56,0.35,0.01}{\textit{#1}}}
\newcommand{\CommentVarTok}[1]{\textcolor[rgb]{0.56,0.35,0.01}{\textbf{\textit{#1}}}}
\newcommand{\ConstantTok}[1]{\textcolor[rgb]{0.00,0.00,0.00}{#1}}
\newcommand{\ControlFlowTok}[1]{\textcolor[rgb]{0.13,0.29,0.53}{\textbf{#1}}}
\newcommand{\DataTypeTok}[1]{\textcolor[rgb]{0.13,0.29,0.53}{#1}}
\newcommand{\DecValTok}[1]{\textcolor[rgb]{0.00,0.00,0.81}{#1}}
\newcommand{\DocumentationTok}[1]{\textcolor[rgb]{0.56,0.35,0.01}{\textbf{\textit{#1}}}}
\newcommand{\ErrorTok}[1]{\textcolor[rgb]{0.64,0.00,0.00}{\textbf{#1}}}
\newcommand{\ExtensionTok}[1]{#1}
\newcommand{\FloatTok}[1]{\textcolor[rgb]{0.00,0.00,0.81}{#1}}
\newcommand{\FunctionTok}[1]{\textcolor[rgb]{0.00,0.00,0.00}{#1}}
\newcommand{\ImportTok}[1]{#1}
\newcommand{\InformationTok}[1]{\textcolor[rgb]{0.56,0.35,0.01}{\textbf{\textit{#1}}}}
\newcommand{\KeywordTok}[1]{\textcolor[rgb]{0.13,0.29,0.53}{\textbf{#1}}}
\newcommand{\NormalTok}[1]{#1}
\newcommand{\OperatorTok}[1]{\textcolor[rgb]{0.81,0.36,0.00}{\textbf{#1}}}
\newcommand{\OtherTok}[1]{\textcolor[rgb]{0.56,0.35,0.01}{#1}}
\newcommand{\PreprocessorTok}[1]{\textcolor[rgb]{0.56,0.35,0.01}{\textit{#1}}}
\newcommand{\RegionMarkerTok}[1]{#1}
\newcommand{\SpecialCharTok}[1]{\textcolor[rgb]{0.00,0.00,0.00}{#1}}
\newcommand{\SpecialStringTok}[1]{\textcolor[rgb]{0.31,0.60,0.02}{#1}}
\newcommand{\StringTok}[1]{\textcolor[rgb]{0.31,0.60,0.02}{#1}}
\newcommand{\VariableTok}[1]{\textcolor[rgb]{0.00,0.00,0.00}{#1}}
\newcommand{\VerbatimStringTok}[1]{\textcolor[rgb]{0.31,0.60,0.02}{#1}}
\newcommand{\WarningTok}[1]{\textcolor[rgb]{0.56,0.35,0.01}{\textbf{\textit{#1}}}}
\usepackage{longtable,booktabs}
% Correct order of tables after \paragraph or \subparagraph
\usepackage{etoolbox}
\makeatletter
\patchcmd\longtable{\par}{\if@noskipsec\mbox{}\fi\par}{}{}
\makeatother
% Allow footnotes in longtable head/foot
\IfFileExists{footnotehyper.sty}{\usepackage{footnotehyper}}{\usepackage{footnote}}
\makesavenoteenv{longtable}
\usepackage{graphicx,grffile}
\makeatletter
\def\maxwidth{\ifdim\Gin@nat@width>\linewidth\linewidth\else\Gin@nat@width\fi}
\def\maxheight{\ifdim\Gin@nat@height>\textheight\textheight\else\Gin@nat@height\fi}
\makeatother
% Scale images if necessary, so that they will not overflow the page
% margins by default, and it is still possible to overwrite the defaults
% using explicit options in \includegraphics[width, height, ...]{}
\setkeys{Gin}{width=\maxwidth,height=\maxheight,keepaspectratio}
% Set default figure placement to htbp
\makeatletter
\def\fps@figure{htbp}
\makeatother
\setlength{\emergencystretch}{3em} % prevent overfull lines
\providecommand{\tightlist}{%
  \setlength{\itemsep}{0pt}\setlength{\parskip}{0pt}}
\setcounter{secnumdepth}{-\maxdimen} % remove section numbering

\title{Week 4 - Homework}
\author{STAT 420, Summer 2020, D. Unger}
\date{}

\begin{document}
\maketitle

\hypertarget{directions}{%
\section{Directions}\label{directions}}

Students are encouraged to work together on homework. However, sharing,
copying or providing any part of a homework solution or code is an
infraction of the University's rules on Academic Integrity. Any
violation will be punished as severely as possible.

\begin{itemize}
\tightlist
\item
  Be sure to remove this section if you use this \texttt{.Rmd} file as a
  template.
\item
  You may leave the questions in your final document.
\end{itemize}

\begin{center}\rule{0.5\linewidth}{0.5pt}\end{center}

\hypertarget{exercise-1-using-lm}{%
\subsection{\texorpdfstring{Exercise 1 (Using
\texttt{lm})}{Exercise 1 (Using lm)}}\label{exercise-1-using-lm}}

For this exercise we will use the data stored in
\href{nutrition-2018.csv}{\texttt{nutrition-2018.csv}}. It contains the
nutritional values per serving size for a large variety of foods as
calculated by the USDA in 2018. It is a cleaned version totaling 5956
observations and is current as of April 2018.

The variables in the dataset are:

\begin{itemize}
\tightlist
\item
  \texttt{ID}
\item
  \texttt{Desc} - short description of food
\item
  \texttt{Water} - in grams
\item
  \texttt{Calories}
\item
  \texttt{Protein} - in grams
\item
  \texttt{Fat} - in grams
\item
  \texttt{Carbs} - carbohydrates, in grams
\item
  \texttt{Fiber} - in grams
\item
  \texttt{Sugar} - in grams
\item
  \texttt{Calcium} - in milligrams
\item
  \texttt{Potassium} - in milligrams
\item
  \texttt{Sodium} - in milligrams
\item
  \texttt{VitaminC} - vitamin C, in milligrams
\item
  \texttt{Chol} - cholesterol, in milligrams
\item
  \texttt{Portion} - description of standard serving size used in
  analysis
\end{itemize}

\textbf{(a)} Fit the following multiple linear regression model in
\texttt{R}. Use \texttt{Calories} as the response and \texttt{Fat},
\texttt{Sugar}, and \texttt{Sodium} as predictors.

\[
Y_i = \beta_0 + \beta_1 x_{i1} + \beta_2 x_{i2} + \beta_3 x_{i3} + \epsilon_i.
\]

Here,

\begin{itemize}
\tightlist
\item
  \(Y_i\) is \texttt{Calories}.
\item
  \(x_{i1}\) is \texttt{Fat}.
\item
  \(x_{i2}\) is \texttt{Sugar}.
\item
  \(x_{i3}\) is \texttt{Sodium}.
\end{itemize}

Use an \(F\)-test to test the significance of the regression. Report the
following:

\begin{itemize}
\tightlist
\item
  The null and alternative hypotheses
\item
  The value of the test statistic
\item
  The p-value of the test
\item
  A statistical decision at \(\alpha = 0.01\)
\item
  A conclusion in the context of the problem
\end{itemize}

When reporting these, you should explicitly state them in your document,
not assume that a reader will find and interpret them from a large block
of \texttt{R} output.

\begin{Shaded}
\begin{Highlighting}[]
\NormalTok{nutrition =}\StringTok{ }\KeywordTok{read.csv}\NormalTok{(}\StringTok{"nutrition-2018.csv"}\NormalTok{)}
\NormalTok{nutrition_lm =}\StringTok{ }\KeywordTok{lm}\NormalTok{(Calories }\OperatorTok{~}\StringTok{ }\NormalTok{Fat}\OperatorTok{+}\NormalTok{Sugar}\OperatorTok{+}\NormalTok{Sodium, }\DataTypeTok{data =}\NormalTok{ nutrition)}
\KeywordTok{summary}\NormalTok{(nutrition_lm)}
\end{Highlighting}
\end{Shaded}

\begin{verbatim}
## 
## Call:
## lm(formula = Calories ~ Fat + Sugar + Sodium, data = nutrition)
## 
## Residuals:
##     Min      1Q  Median      3Q     Max 
## -339.41  -64.82   -9.42   28.12  293.54 
## 
## Coefficients:
##              Estimate Std. Error t value Pr(>|t|)    
## (Intercept) 1.005e+02  1.409e+00  71.310  < 2e-16 ***
## Fat         8.483e+00  6.456e-02 131.394  < 2e-16 ***
## Sugar       3.901e+00  7.140e-02  54.627  < 2e-16 ***
## Sodium      6.165e-03  1.030e-03   5.983 2.31e-09 ***
## ---
## Signif. codes:  0 '***' 0.001 '**' 0.01 '*' 0.05 '.' 0.1 ' ' 1
## 
## Residual standard error: 80.85 on 5952 degrees of freedom
## Multiple R-squared:  0.7686, Adjusted R-squared:  0.7685 
## F-statistic:  6591 on 3 and 5952 DF,  p-value: < 2.2e-16
\end{verbatim}

\(Ho: \beta_1 = \beta_2 = \beta_3 = 0\) and \(H1:\) atleast one of
\(\beta_j \neq 0\) Value of F statistic is 6591

p-value of the test is almost 0. Very very small value (\textless{}
2.2e-16).

Satistical decision at \(\alpha = 0.01\) is to Reject \(Ho\)

Conclusion in the context of the problem is that there EXISTS a linear
relationship between Calories and atleast some of fat, sugar and sodium.

\textbf{(b)} Output only the estimated regression coefficients.
Interpret all \(\hat{\beta}_j\) coefficients in the context of the
problem.

\begin{Shaded}
\begin{Highlighting}[]
\KeywordTok{coef}\NormalTok{(nutrition_lm)}
\end{Highlighting}
\end{Shaded}

\begin{verbatim}
##  (Intercept)          Fat        Sugar       Sodium 
## 1.004561e+02 8.483289e+00 3.900517e+00 6.165246e-03
\end{verbatim}

\(\hat{\beta}_0\) = 100.4561 is the estimated calories value of a food
when there is 0gm fat, 0gm sugar and 0mg Sodium

\(\hat{\beta}_1\) = 8.483289 is the estimated change in mean Calories
value of a food when there is an increase of 1gm fat with certain Sugar
and Sodium content in the food

\(\hat{\beta}_2\) = 3.900517 is the estimated change in mean Calories
value of a food when there is an increase of 1gm sugar with certain fat
and Sodium in the food.

\(\hat{\beta}_3\) - 0.006165246 is the estimated change in mean Calories
value of a food when there is an increase of 1mg of Sodium with certain
fat and sugare in the food

\textbf{(c)} Use your model to predict the number of \texttt{Calories}
in a Big Mac. According to
\href{https://www.mcdonalds.com/us/en-us/about-our-food/nutrition-calculator.html}{McDonald's
publicized nutrition facts}, the Big Mac contains 30g of fat, 9g of
sugar, and 1010mg of sodium.

\begin{Shaded}
\begin{Highlighting}[]
\NormalTok{big_mac_query =}\StringTok{ }\KeywordTok{data.frame}\NormalTok{(}\DataTypeTok{Fat =} \DecValTok{30}\NormalTok{, }\DataTypeTok{Sugar =} \DecValTok{9}\NormalTok{, }\DataTypeTok{Sodium =} \DecValTok{1010}\NormalTok{)}
\KeywordTok{predict}\NormalTok{(nutrition_lm, }\DataTypeTok{newdata =}\NormalTok{ big_mac_query)}
\end{Highlighting}
\end{Shaded}

\begin{verbatim}
##        1 
## 396.2863
\end{verbatim}

Nuber of Calories is 396.2863

\textbf{(d)} Calculate the standard deviation, \(s_y\), for the observed
values in the Calories variable. Report the value of \(s_e\) from your
multiple regression model. Interpret both estimates in the context of
this problem.

\begin{Shaded}
\begin{Highlighting}[]
\NormalTok{(}\DataTypeTok{s_y =} \KeywordTok{sd}\NormalTok{(nutrition}\OperatorTok{$}\NormalTok{Calories))}
\end{Highlighting}
\end{Shaded}

\begin{verbatim}
## [1] 168.05
\end{verbatim}

\begin{Shaded}
\begin{Highlighting}[]
\NormalTok{(}\DataTypeTok{s_e =} \KeywordTok{summary}\NormalTok{(nutrition_lm)}\OperatorTok{$}\NormalTok{sigma)}
\end{Highlighting}
\end{Shaded}

\begin{verbatim}
## [1] 80.8543
\end{verbatim}

\(s_y\) = 168.05 gives us an estimate of the variations in the Calories
value around its Mean Value.

\(s_e\) = 80.8543 gives us an estimate of the variations in the
residuals of the model, especially how the observed Calorie value vares
from it's fitted value by considering the predictors value.

\textbf{(e)} Report the value of \(R^2\) for the model. Interpret its
meaning in the context of the problem.

\begin{Shaded}
\begin{Highlighting}[]
\CommentTok{#names(summary(nutrition_lm))}
\KeywordTok{summary}\NormalTok{(nutrition_lm)}\OperatorTok{$}\NormalTok{r.squared}
\end{Highlighting}
\end{Shaded}

\begin{verbatim}
## [1] 0.7686281
\end{verbatim}

\(R^2\) = 0.7686281 explains that 76.8\% of the observed variations in
Calories can be explained by a linear relationsip with fat, sugar and
sodium in the food.

\textbf{(f)} Calculate a 90\% confidence interval for \(\beta_2\). Give
an interpretation of the interval in the context of the problem.

\begin{Shaded}
\begin{Highlighting}[]
\KeywordTok{confint}\NormalTok{(nutrition_lm, }\DataTypeTok{level =} \FloatTok{0.90}\NormalTok{)[}\DecValTok{3}\NormalTok{,]}
\end{Highlighting}
\end{Shaded}

\begin{verbatim}
##      5 %     95 % 
## 3.783051 4.017983
\end{verbatim}

We are 90\% confident that the mean Calorie for an increase of 1 gm of
Sugar is in the range of 3.783051 and 4.01798 for certain values of fat
and sodium

\textbf{(g)} Calculate a 95\% confidence interval for \(\beta_0\). Give
an interpretation of the interval in the context of the problem.

\begin{Shaded}
\begin{Highlighting}[]
\KeywordTok{confint}\NormalTok{(nutrition_lm, }\DataTypeTok{level =} \FloatTok{0.95}\NormalTok{)[}\DecValTok{1}\NormalTok{,]}
\end{Highlighting}
\end{Shaded}

\begin{verbatim}
##     2.5 %    97.5 % 
##  97.69443 103.21768
\end{verbatim}

We are 95\% confidence that the true mean Calorie for food with 0gm of
fat, 0gm of sugar and 0mg of Sodium is in the range of 97.69443 and
103.21768

\textbf{(h)} Use a 99\% confidence interval to estimate the mean Calorie
content of a food with 23g of fat, 0g of sugar, and 400mg of sodium,
which is true of a large order of McDonald's french fries. Interpret the
interval in context.

\begin{Shaded}
\begin{Highlighting}[]
\NormalTok{large_fries_query =}\StringTok{ }\KeywordTok{data.frame}\NormalTok{(}\DataTypeTok{Fat =} \DecValTok{23}\NormalTok{, }\DataTypeTok{Sugar =} \DecValTok{0}\NormalTok{, }\DataTypeTok{Sodium =} \DecValTok{400}\NormalTok{)}
\KeywordTok{predict}\NormalTok{(nutrition_lm, }\DataTypeTok{newdata =}\NormalTok{ large_fries_query, }\DataTypeTok{interval =} \StringTok{"confidence"}\NormalTok{, }\DataTypeTok{level =} \FloatTok{0.99}\NormalTok{)}
\end{Highlighting}
\end{Shaded}

\begin{verbatim}
##        fit      lwr      upr
## 1 298.0378 294.3532 301.7224
\end{verbatim}

We are 99\% confident that the true mean calorie value for food with 23g
of fat, 0g of sugare and 400mg of Sodium is in the range of 294.3532 and
301.7224

\textbf{(i)} Use a 99\% prediction interval to predict the Calorie
content of a Crunchwrap Supreme, which has 21g of fat, 6g of sugar, and
1200mg of sodium according to
\href{https://www.tacobell.com/nutrition/info}{Taco Bell's publicized
nutrition information}. Interpret the interval in context.

\begin{Shaded}
\begin{Highlighting}[]
\NormalTok{crunchwrapsupreme_query =}\StringTok{ }\KeywordTok{data.frame}\NormalTok{(}\DataTypeTok{Fat =} \DecValTok{21}\NormalTok{, }\DataTypeTok{Sugar =} \DecValTok{6}\NormalTok{, }\DataTypeTok{Sodium =} \DecValTok{1200}\NormalTok{)}
\KeywordTok{predict}\NormalTok{(nutrition_lm, }\DataTypeTok{newdata =}\NormalTok{ crunchwrapsupreme_query, }\DataTypeTok{interval =} \StringTok{"prediction"}\NormalTok{, }\DataTypeTok{level =} \FloatTok{.99}\NormalTok{)}
\end{Highlighting}
\end{Shaded}

\begin{verbatim}
##        fit      lwr      upr
## 1 309.4065 101.0345 517.7786
\end{verbatim}

We are 99\% confident that the true mean calorie value for food with 21g
of fat, 6g of sugar and 1200mg of sodium is in the range of 101.0345 and
517.7786

\begin{center}\rule{0.5\linewidth}{0.5pt}\end{center}

\hypertarget{exercise-2-more-lm-for-multiple-regression}{%
\subsection{\texorpdfstring{Exercise 2 (More \texttt{lm} for Multiple
Regression)}{Exercise 2 (More lm for Multiple Regression)}}\label{exercise-2-more-lm-for-multiple-regression}}

For this exercise we will use the data stored in
\href{goalies.csv}{\texttt{goalies.csv}}. It contains career data for
462 players in the National Hockey League who played goaltender at some
point up to and including the 2014-2015 season. The variables in the
dataset are:

\begin{itemize}
\tightlist
\item
  \texttt{W} - Wins
\item
  \texttt{GA} - Goals Against
\item
  \texttt{SA} - Shots Against
\item
  \texttt{SV} - Saves
\item
  \texttt{SV\_PCT} - Save Percentage
\item
  \texttt{GAA} - Goals Against Average
\item
  \texttt{SO} - Shutouts
\item
  \texttt{MIN} - Minutes
\item
  \texttt{PIM} - Penalties in Minutes
\end{itemize}

For this exercise we will consider three models, each with Wins as the
response. The predictors for these models are:

\begin{itemize}
\tightlist
\item
  Model 1: Goals Against, Saves
\item
  Model 2: Goals Against, Saves, Shots Against, Minutes, Shutouts
\item
  Model 3: All Available
\end{itemize}

\begin{Shaded}
\begin{Highlighting}[]
\NormalTok{goalies =}\StringTok{ }\KeywordTok{read.csv}\NormalTok{(}\StringTok{"goalies.csv"}\NormalTok{)}
\NormalTok{glm_}\DecValTok{1}\NormalTok{ =}\StringTok{ }\KeywordTok{lm}\NormalTok{(W }\OperatorTok{~}\StringTok{ }\NormalTok{GA }\OperatorTok{+}\StringTok{ }\NormalTok{SV, }\DataTypeTok{data =}\NormalTok{ goalies)}
\NormalTok{glm_}\DecValTok{2}\NormalTok{ =}\StringTok{ }\KeywordTok{lm}\NormalTok{(W }\OperatorTok{~}\StringTok{ }\NormalTok{GA }\OperatorTok{+}\StringTok{ }\NormalTok{SV }\OperatorTok{+}\StringTok{ }\NormalTok{SA }\OperatorTok{+}\StringTok{ }\NormalTok{MIN }\OperatorTok{+}\StringTok{ }\NormalTok{SO, }\DataTypeTok{data =}\NormalTok{ goalies)}
\NormalTok{glm_}\DecValTok{3}\NormalTok{ =}\StringTok{ }\KeywordTok{lm}\NormalTok{(W }\OperatorTok{~}\StringTok{ }\NormalTok{., }\DataTypeTok{data =}\NormalTok{ goalies)}
\end{Highlighting}
\end{Shaded}

\textbf{(a)} Use an \(F\)-test to compares Models 1 and 2. Report the
following:

\begin{itemize}
\tightlist
\item
  The null hypothesis
\item
  The value of the test statistic
\item
  The p-value of the test
\item
  A statistical decision at \(\alpha = 0.05\)
\item
  The model you prefer
\end{itemize}

\begin{Shaded}
\begin{Highlighting}[]
\KeywordTok{anova}\NormalTok{(glm_}\DecValTok{1}\NormalTok{, glm_}\DecValTok{2}\NormalTok{)}
\end{Highlighting}
\end{Shaded}

\begin{verbatim}
## Analysis of Variance Table
## 
## Model 1: W ~ GA + SV
## Model 2: W ~ GA + SV + SA + MIN + SO
##   Res.Df    RSS Df Sum of Sq      F    Pr(>F)    
## 1    459 294757                                  
## 2    456  72899  3    221858 462.59 < 2.2e-16 ***
## ---
## Signif. codes:  0 '***' 0.001 '**' 0.01 '*' 0.05 '.' 0.1 ' ' 1
\end{verbatim}

\(Ho\) : \(\beta_{\texttt{SA}}\) = \(\beta_{\texttt{SV}}\) =
\(\beta_{\texttt{SO}}\) = 0

The value of the Test Statistics is F = 462.59

The p-value of the test is 2.2e-16

A statistical decision at \(\alpha = 0.05\) is Reject \(Ho\)

Model Preference: Model 2

\textbf{(b)} Use an \(F\)-test to compare Model 3 to your preferred
model from part \textbf{(a)}. Report the following:

\begin{itemize}
\tightlist
\item
  The null hypothesis
\item
  The value of the test statistic
\item
  The p-value of the test
\item
  A statistical decision at \(\alpha = 0.05\)
\item
  The model you prefer
\end{itemize}

\begin{Shaded}
\begin{Highlighting}[]
\KeywordTok{anova}\NormalTok{(glm_}\DecValTok{2}\NormalTok{, glm_}\DecValTok{3}\NormalTok{)}
\end{Highlighting}
\end{Shaded}

\begin{verbatim}
## Analysis of Variance Table
## 
## Model 1: W ~ GA + SV + SA + MIN + SO
## Model 2: W ~ GA + SA + SV + SV_PCT + GAA + SO + MIN + PIM
##   Res.Df   RSS Df Sum of Sq     F   Pr(>F)   
## 1    456 72899                               
## 2    453 70994  3    1905.1 4.052 0.007353 **
## ---
## Signif. codes:  0 '***' 0.001 '**' 0.01 '*' 0.05 '.' 0.1 ' ' 1
\end{verbatim}

\(Ho\) : \(\beta_{\texttt{SVPCT}}\) = \(\beta_{\texttt{GAA}}\) =
\(\beta_{\texttt{PIM}}\) = 0

The value of the Test Statistics is F = 4.052

The p-value of the test is 0.007353

A statistical decision at \(\alpha = 0.05\) is Reject \(Ho\)

Model Preference: Model 3

\textbf{(c)} Use a \(t\)-test to test
\(H_0: \beta_{\texttt{SV}} = 0 \ \text{vs} \ H_1: \beta_{\texttt{SV}} \neq 0\)
for the model you preferred in part \textbf{(b)}. Report the following:

\begin{itemize}
\tightlist
\item
  The value of the test statistic
\item
  The p-value of the test
\item
  A statistical decision at \(\alpha = 0.05\)
\end{itemize}

\begin{Shaded}
\begin{Highlighting}[]
\KeywordTok{summary}\NormalTok{(glm_}\DecValTok{3}\NormalTok{)}
\end{Highlighting}
\end{Shaded}

\begin{verbatim}
## 
## Call:
## lm(formula = W ~ ., data = goalies)
## 
## Residuals:
##     Min      1Q  Median      3Q     Max 
## -51.204  -3.126   0.935   2.835  64.078 
## 
## Coefficients:
##               Estimate Std. Error t value Pr(>|t|)    
## (Intercept)  5.2651619 16.8181423   0.313 0.754376    
## GA          -0.1132805  0.0148085  -7.650 1.22e-13 ***
## SA           0.0516385  0.0135565   3.809 0.000159 ***
## SV          -0.0582151  0.0150905  -3.858 0.000131 ***
## SV_PCT      -8.0475191 17.6600154  -0.456 0.648830    
## GAA         -0.0496006  0.4821957  -0.103 0.918116    
## SO           0.4599359  0.1989567   2.312 0.021240 *  
## MIN          0.0131790  0.0009504  13.867  < 2e-16 ***
## PIM          0.0468422  0.0136373   3.435 0.000647 ***
## ---
## Signif. codes:  0 '***' 0.001 '**' 0.01 '*' 0.05 '.' 0.1 ' ' 1
## 
## Residual standard error: 12.52 on 453 degrees of freedom
## Multiple R-squared:  0.9858, Adjusted R-squared:  0.9856 
## F-statistic:  3938 on 8 and 453 DF,  p-value: < 2.2e-16
\end{verbatim}

The value of the test statistic is -3.858

The p-value of the test is 0.000131

A statistical decision at \(\alpha = 0.05\) is Reject \(Ho\)

\begin{center}\rule{0.5\linewidth}{0.5pt}\end{center}

\hypertarget{exercise-3-regression-without-lm}{%
\subsection{\texorpdfstring{Exercise 3 (Regression without
\texttt{lm})}{Exercise 3 (Regression without lm)}}\label{exercise-3-regression-without-lm}}

For this exercise we will once again use the \texttt{Ozone} data from
the \texttt{mlbench} package. The goal of this exercise is to fit a
model with \texttt{ozone} as the response and the remaining variables as
predictors.

\begin{Shaded}
\begin{Highlighting}[]
\KeywordTok{data}\NormalTok{(Ozone, }\DataTypeTok{package =} \StringTok{"mlbench"}\NormalTok{)}
\NormalTok{Ozone =}\StringTok{ }\NormalTok{Ozone[, }\KeywordTok{c}\NormalTok{(}\DecValTok{4}\NormalTok{, }\DecValTok{6}\NormalTok{, }\DecValTok{7}\NormalTok{, }\DecValTok{8}\NormalTok{)]}
\KeywordTok{colnames}\NormalTok{(Ozone) =}\StringTok{ }\KeywordTok{c}\NormalTok{(}\StringTok{"ozone"}\NormalTok{, }\StringTok{"wind"}\NormalTok{, }\StringTok{"humidity"}\NormalTok{, }\StringTok{"temp"}\NormalTok{)}
\NormalTok{Ozone =}\StringTok{ }\NormalTok{Ozone[}\KeywordTok{complete.cases}\NormalTok{(Ozone), ]}
\end{Highlighting}
\end{Shaded}

\textbf{(a)} Obtain the estimated regression coefficients
\textbf{without} the use of \texttt{lm()} or any other built-in
functions for regression. That is, you should use only matrix
operations. Store the results in a vector \texttt{beta\_hat\_no\_lm}. To
ensure this is a vector, you may need to use \texttt{as.vector()}.
Return this vector as well as the results of
\texttt{sum(beta\_hat\_no\_lm\ \^{}\ 2)}.

\begin{Shaded}
\begin{Highlighting}[]
\NormalTok{n =}\StringTok{ }\KeywordTok{nrow}\NormalTok{(Ozone)}
\NormalTok{X =}\StringTok{ }\KeywordTok{cbind}\NormalTok{(}\KeywordTok{rep}\NormalTok{(}\DecValTok{1}\NormalTok{,n), }\KeywordTok{as.matrix}\NormalTok{(}\KeywordTok{subset}\NormalTok{(Ozone, }\DataTypeTok{select =} \KeywordTok{c}\NormalTok{(}\StringTok{"wind"}\NormalTok{, }\StringTok{"humidity"}\NormalTok{, }\StringTok{"temp"}\NormalTok{))))}
\NormalTok{y =}\StringTok{ }\NormalTok{Ozone}\OperatorTok{$}\NormalTok{ozone}

\NormalTok{beta_hat_no_lm =}\StringTok{ }\KeywordTok{solve}\NormalTok{(}\KeywordTok{t}\NormalTok{(X) }\OperatorTok\StringTok{ }\NormalTok{X) }\OperatorTok\StringTok{ }\KeywordTok{t}\NormalTok{(X) }\OperatorTok\StringTok{ }\NormalTok{y}
\NormalTok{beta_hat_no_lm =}\StringTok{ }\KeywordTok{as.vector}\NormalTok{(beta_hat_no_lm)}
\NormalTok{beta_hat_no_lm}
\end{Highlighting}
\end{Shaded}

\begin{verbatim}
## [1] -16.38178539  -0.18594444   0.08340014   0.38984294
\end{verbatim}

\begin{Shaded}
\begin{Highlighting}[]
\KeywordTok{sum}\NormalTok{(beta_hat_no_lm }\OperatorTok{^}\StringTok{ }\DecValTok{2}\NormalTok{)}
\end{Highlighting}
\end{Shaded}

\begin{verbatim}
## [1] 268.5564
\end{verbatim}

\textbf{(b)} Obtain the estimated regression coefficients \textbf{with}
the use of \texttt{lm()}. Store the results in a vector
\texttt{beta\_hat\_lm}. To ensure this is a vector, you may need to use
\texttt{as.vector()}. Return this vector as well as the results of
\texttt{sum(beta\_hat\_lm\ \^{}\ 2)}.

\begin{Shaded}
\begin{Highlighting}[]
\NormalTok{ozone_lm =}\StringTok{ }\KeywordTok{lm}\NormalTok{(ozone }\OperatorTok{~}\StringTok{ }\NormalTok{., }\DataTypeTok{data =}\NormalTok{ Ozone)}
\NormalTok{beta_hat_lm =}\StringTok{ }\KeywordTok{as.vector}\NormalTok{(}\KeywordTok{coef}\NormalTok{(ozone_lm))}
\NormalTok{beta_hat_lm}
\end{Highlighting}
\end{Shaded}

\begin{verbatim}
## [1] -16.38178539  -0.18594444   0.08340014   0.38984294
\end{verbatim}

\begin{Shaded}
\begin{Highlighting}[]
\KeywordTok{sum}\NormalTok{(beta_hat_lm }\OperatorTok{^}\StringTok{ }\DecValTok{2}\NormalTok{)}
\end{Highlighting}
\end{Shaded}

\begin{verbatim}
## [1] 268.5564
\end{verbatim}

\textbf{(c)} Use the \texttt{all.equal()} function to verify that the
results are the same. You may need to remove the names of one of the
vectors. The \texttt{as.vector()} function will do this as a side
effect, or you can directly use \texttt{unname()}.

\begin{Shaded}
\begin{Highlighting}[]
\KeywordTok{all.equal}\NormalTok{(beta_hat_no_lm, beta_hat_lm)}
\end{Highlighting}
\end{Shaded}

\begin{verbatim}
## [1] TRUE
\end{verbatim}

\textbf{(d)} Calculate \(s_e\) without the use of \texttt{lm()}. That
is, continue with your results from \textbf{(a)} and perform additional
matrix operations to obtain the result. Output this result. Also, verify
that this result is the same as the result obtained from \texttt{lm()}.

\begin{Shaded}
\begin{Highlighting}[]
\CommentTok{#summary(ozone_lm)$sigma}

\NormalTok{y_hat =}\StringTok{ }\NormalTok{X }\OperatorTok\StringTok{ }\KeywordTok{solve}\NormalTok{(}\KeywordTok{t}\NormalTok{(X) }\OperatorTok\StringTok{ }\NormalTok{X) }\OperatorTok\StringTok{ }\KeywordTok{t}\NormalTok{(X) }\OperatorTok\StringTok{ }\NormalTok{y}
\NormalTok{e =}\StringTok{ }\NormalTok{y }\OperatorTok{-}\StringTok{ }\NormalTok{y_hat}
\NormalTok{(}\DataTypeTok{s_e =} \KeywordTok{sqrt}\NormalTok{(}\KeywordTok{t}\NormalTok{(e) }\OperatorTok\StringTok{ }\NormalTok{e}\OperatorTok{/}\StringTok{ }\NormalTok{(n }\OperatorTok{-}\StringTok{ }\KeywordTok{length}\NormalTok{(beta_hat_no_lm)))[}\DecValTok{1}\NormalTok{,])}
\end{Highlighting}
\end{Shaded}

\begin{verbatim}
## [1] 4.806115
\end{verbatim}

\begin{Shaded}
\begin{Highlighting}[]
\KeywordTok{all.equal}\NormalTok{(}\KeywordTok{summary}\NormalTok{(ozone_lm)}\OperatorTok{$}\NormalTok{sigma, s_e)}
\end{Highlighting}
\end{Shaded}

\begin{verbatim}
## [1] TRUE
\end{verbatim}

\textbf{(e)} Calculate \(R^2\) without the use of \texttt{lm()}. That
is, continue with your results from \textbf{(a)} and \textbf{(d)}, and
perform additional operations to obtain the result. Output this result.
Also, verify that this result is the same as the result obtained from
\texttt{lm()}.

\begin{Shaded}
\begin{Highlighting}[]
\NormalTok{sse =}\StringTok{ }\KeywordTok{sum}\NormalTok{(}\KeywordTok{t}\NormalTok{(y}\OperatorTok{-}\NormalTok{y_hat) }\OperatorTok\StringTok{ }\NormalTok{(y}\OperatorTok{-}\NormalTok{y_hat))}
\NormalTok{sst =}\StringTok{ }\KeywordTok{sum}\NormalTok{(}\KeywordTok{t}\NormalTok{(y}\OperatorTok{-}\KeywordTok{mean}\NormalTok{(y)) }\OperatorTok\StringTok{ }\NormalTok{(y}\OperatorTok{-}\KeywordTok{mean}\NormalTok{(y)))}

\NormalTok{(}\DataTypeTok{r2 =} \DecValTok{1} \OperatorTok{-}\StringTok{ }\NormalTok{sse}\OperatorTok{/}\NormalTok{sst)}
\end{Highlighting}
\end{Shaded}

\begin{verbatim}
## [1] 0.6398887
\end{verbatim}

\begin{Shaded}
\begin{Highlighting}[]
\KeywordTok{all.equal}\NormalTok{(r2, }\KeywordTok{summary}\NormalTok{(ozone_lm)}\OperatorTok{$}\NormalTok{r.squared)  }
\end{Highlighting}
\end{Shaded}

\begin{verbatim}
## [1] TRUE
\end{verbatim}

\begin{center}\rule{0.5\linewidth}{0.5pt}\end{center}

\hypertarget{exercise-4-regression-for-prediction}{%
\subsection{Exercise 4 (Regression for
Prediction)}\label{exercise-4-regression-for-prediction}}

For this exercise use the \texttt{Auto} dataset from the \texttt{ISLR}
package. Use \texttt{?Auto} to learn about the dataset. The goal of this
exercise is to find a model that is useful for \textbf{predicting} the
response \texttt{mpg}. We remove the \texttt{name} variable as it is not
useful for this analysis. (Also, this is an easier to load version of
data from the textbook.)

\begin{Shaded}
\begin{Highlighting}[]
\CommentTok{# load required package, remove "name" variable}
\KeywordTok{library}\NormalTok{(ISLR)}
\NormalTok{Auto =}\StringTok{ }\KeywordTok{subset}\NormalTok{(Auto, }\DataTypeTok{select =} \OperatorTok{-}\KeywordTok{c}\NormalTok{(name)) }\CommentTok{#Error in eval(substitute(select), nl, parent.frame()) : object 'name' not found}
\end{Highlighting}
\end{Shaded}

When evaluating a model for prediction, we often look at RMSE. However,
if we both fit the model with all the data as well as evaluate RMSE
using all the data, we're essentially cheating. We'd like to use RMSE as
a measure of how well the model will predict on \emph{unseen} data. If
you haven't already noticed, the way we had been using RMSE resulted in
RMSE decreasing as models became larger.

To correct for this, we will only use a portion of the data to fit the
model, and then we will use leftover data to evaluate the model. We will
call these datasets \textbf{train} (for fitting) and \textbf{test} (for
evaluating). The definition of RMSE will stay the same

\[
\text{RMSE}(\text{model, data}) = \sqrt{\frac{1}{n} \sum_{i = 1}^{n}(y_i - \hat{y}_i)^2}
\]

where

\begin{itemize}
\tightlist
\item
  \(y_i\) are the actual values of the response for the given data.
\item
  \(\hat{y}_i\) are the predicted values using the fitted model and the
  predictors from the data.
\end{itemize}

However, we will now evaluate it on both the \textbf{train} set and the
\textbf{test} set separately. So each model you fit will have a
\textbf{train} RMSE and a \textbf{test} RMSE. When calculating
\textbf{test} RMSE, the predicted values will be found by predicting the
response using the \textbf{test} data with the model fit using the
\textbf{train} data. \emph{\textbf{Test} data should never be used to
fit a model.}

\begin{itemize}
\tightlist
\item
  Train RMSE: Model fit with \emph{train} data. Evaluate on
  \textbf{train} data.
\item
  Test RMSE: Model fit with \emph{train} data. Evaluate on \textbf{test}
  data.
\end{itemize}

Set a seed of \texttt{11}, and then split the \texttt{Auto} data into
two datasets, one called \texttt{auto\_trn} and one called
\texttt{auto\_tst}. The \texttt{auto\_trn} data frame should contain 292
randomly chosen observations. The \texttt{auto\_tst} data will contain
the remaining observations. Hint: consider the following code:

\begin{Shaded}
\begin{Highlighting}[]
\KeywordTok{set.seed}\NormalTok{(}\DecValTok{11}\NormalTok{)}
\NormalTok{auto_trn_idx =}\StringTok{ }\KeywordTok{sample}\NormalTok{(}\DecValTok{1}\OperatorTok{:}\KeywordTok{nrow}\NormalTok{(Auto), }\DecValTok{292}\NormalTok{)}
\KeywordTok{names}\NormalTok{(Auto)}
\end{Highlighting}
\end{Shaded}

Fit a total of five models using the training data.

\begin{itemize}
\tightlist
\item
  One must use all possible predictors.
\item
  One must use only \texttt{displacement} as a predictor.
\item
  The remaining three you can pick to be anything you like. One of these
  should be the \emph{best} of the five for predicting the response.
\end{itemize}

For each model report the \textbf{train} and \textbf{test} RMSE. Arrange
your results in a well-formatted markdown table. Argue that one of your
models is the best for predicting the response.

\begin{Shaded}
\begin{Highlighting}[]
\NormalTok{rmse =}\StringTok{ }\ControlFlowTok{function}\NormalTok{(actual, predicted) \{}
  \KeywordTok{sqrt}\NormalTok{(}\KeywordTok{mean}\NormalTok{((actual }\OperatorTok{-}\StringTok{ }\NormalTok{predicted)}\OperatorTok{^}\DecValTok{2}\NormalTok{)) }
\NormalTok{\}}

\KeywordTok{set.seed}\NormalTok{(}\DecValTok{11}\NormalTok{)}
\NormalTok{auto_trn_idx =}\StringTok{ }\KeywordTok{sample}\NormalTok{(}\DecValTok{1}\OperatorTok{:}\KeywordTok{nrow}\NormalTok{(Auto), }\DecValTok{292}\NormalTok{)}
\NormalTok{auto_trn =}\StringTok{ }\NormalTok{Auto[auto_trn_idx, ]}
\NormalTok{auto_tst =}\StringTok{ }\NormalTok{Auto[}\OperatorTok{-}\NormalTok{auto_trn_idx, ]}

\NormalTok{auto_}\DecValTok{1}\NormalTok{ =}\StringTok{ }\KeywordTok{lm}\NormalTok{(mpg }\OperatorTok{~}\StringTok{ }\NormalTok{., }\DataTypeTok{data =}\NormalTok{ auto_trn)}
\NormalTok{auto_}\DecValTok{2}\NormalTok{ =}\StringTok{ }\KeywordTok{lm}\NormalTok{(mpg }\OperatorTok{~}\StringTok{ }\NormalTok{displacement, }\DataTypeTok{data =}\NormalTok{ auto_trn)}
\NormalTok{auto_}\DecValTok{3}\NormalTok{ =}\StringTok{ }\KeywordTok{lm}\NormalTok{(mpg }\OperatorTok{~}\StringTok{ }\NormalTok{cylinders }\OperatorTok{+}\StringTok{ }\NormalTok{displacement }\OperatorTok{+}\StringTok{ }\NormalTok{weight }\OperatorTok{+}\StringTok{ }\NormalTok{year }\OperatorTok{+}\StringTok{ }\NormalTok{origin, }\DataTypeTok{data =}\NormalTok{ auto_trn)}
\NormalTok{auto_}\DecValTok{4}\NormalTok{ =}\StringTok{ }\KeywordTok{lm}\NormalTok{(mpg }\OperatorTok{~}\StringTok{ }\NormalTok{cylinders }\OperatorTok{+}\StringTok{ }\NormalTok{displacement, }\DataTypeTok{data =}\NormalTok{ auto_trn)}
\NormalTok{auto_}\DecValTok{5}\NormalTok{ =}\StringTok{ }\KeywordTok{lm}\NormalTok{(mpg }\OperatorTok{~}\StringTok{ }\NormalTok{displacement }\OperatorTok{+}\StringTok{ }\NormalTok{horsepower }\OperatorTok{+}\StringTok{ }\NormalTok{weight, }\DataTypeTok{data =}\NormalTok{ auto_trn)}

\NormalTok{trn_res =}\StringTok{ }\KeywordTok{c}\NormalTok{(}\KeywordTok{rmse}\NormalTok{(auto_trn}\OperatorTok{$}\NormalTok{mpg, }\KeywordTok{predict}\NormalTok{(auto_}\DecValTok{1}\NormalTok{, auto_trn)),}
            \KeywordTok{rmse}\NormalTok{(auto_trn}\OperatorTok{$}\NormalTok{mpg, }\KeywordTok{predict}\NormalTok{(auto_}\DecValTok{2}\NormalTok{, auto_trn)),}
            \KeywordTok{rmse}\NormalTok{(auto_trn}\OperatorTok{$}\NormalTok{mpg, }\KeywordTok{predict}\NormalTok{(auto_}\DecValTok{3}\NormalTok{, auto_trn)),}
            \KeywordTok{rmse}\NormalTok{(auto_trn}\OperatorTok{$}\NormalTok{mpg, }\KeywordTok{predict}\NormalTok{(auto_}\DecValTok{4}\NormalTok{, auto_trn)),}
            \KeywordTok{rmse}\NormalTok{(auto_trn}\OperatorTok{$}\NormalTok{mpg, }\KeywordTok{predict}\NormalTok{(auto_}\DecValTok{5}\NormalTok{, auto_trn)))}

\NormalTok{tst_res =}\StringTok{ }\KeywordTok{c}\NormalTok{(}\KeywordTok{rmse}\NormalTok{(auto_trn}\OperatorTok{$}\NormalTok{mpg, }\KeywordTok{predict}\NormalTok{(auto_}\DecValTok{1}\NormalTok{, auto_tst)),}
            \KeywordTok{rmse}\NormalTok{(auto_trn}\OperatorTok{$}\NormalTok{mpg, }\KeywordTok{predict}\NormalTok{(auto_}\DecValTok{2}\NormalTok{, auto_tst)),}
            \KeywordTok{rmse}\NormalTok{(auto_trn}\OperatorTok{$}\NormalTok{mpg, }\KeywordTok{predict}\NormalTok{(auto_}\DecValTok{3}\NormalTok{, auto_tst)),}
            \KeywordTok{rmse}\NormalTok{(auto_trn}\OperatorTok{$}\NormalTok{mpg, }\KeywordTok{predict}\NormalTok{(auto_}\DecValTok{4}\NormalTok{, auto_tst)),}
            \KeywordTok{rmse}\NormalTok{(auto_trn}\OperatorTok{$}\NormalTok{mpg, }\KeywordTok{predict}\NormalTok{(auto_}\DecValTok{5}\NormalTok{, auto_tst)))}
\end{Highlighting}
\end{Shaded}

\begin{verbatim}
## Warning in actual - predicted: longer object length is not a multiple of shorter
## object length

## Warning in actual - predicted: longer object length is not a multiple of shorter
## object length

## Warning in actual - predicted: longer object length is not a multiple of shorter
## object length

## Warning in actual - predicted: longer object length is not a multiple of shorter
## object length

## Warning in actual - predicted: longer object length is not a multiple of shorter
## object length
\end{verbatim}

\begin{Shaded}
\begin{Highlighting}[]
\NormalTok{test_summary =}\StringTok{ }\KeywordTok{data.frame}\NormalTok{(}\DataTypeTok{Model =} \KeywordTok{c}\NormalTok{(}\StringTok{"auto_1"}\NormalTok{, }\StringTok{"auto_2"}\NormalTok{, }\StringTok{"auto_3"}\NormalTok{, }\StringTok{"auto_4"}\NormalTok{, }\StringTok{"auto_5"}\NormalTok{),}
                          \DataTypeTok{Trained_RMSE =}\NormalTok{ trn_res, }
                          \DataTypeTok{Tested_RMSE =}\NormalTok{ tst_res)}
\KeywordTok{colnames}\NormalTok{(test_summary) =}\StringTok{ }\KeywordTok{c}\NormalTok{(}\StringTok{"Train Model"}\NormalTok{, }\StringTok{"Train RMSE"}\NormalTok{, }\StringTok{"Test RMSE"}\NormalTok{)}

\KeywordTok{library}\NormalTok{(knitr)}
\KeywordTok{kable}\NormalTok{(test_summary)}
\end{Highlighting}
\end{Shaded}

\begin{longtable}[]{@{}lrr@{}}
\toprule
Train Model & Train RMSE & Test RMSE\tabularnewline
\midrule
\endhead
auto\_1 & 3.223650 & 10.58236\tabularnewline
auto\_2 & 4.553319 & 10.64192\tabularnewline
auto\_3 & 3.245324 & 10.54884\tabularnewline
auto\_4 & 4.539067 & 10.59205\tabularnewline
auto\_5 & 4.141059 & 10.37919\tabularnewline
\bottomrule
\end{longtable}

Based on the above results, auto\_4 model is the best model for
predicting, since it has lowest Test\_RMSE value of 3.566250 with
``cylinders + displacement + weight + year + origin'' predictors.

\begin{center}\rule{0.5\linewidth}{0.5pt}\end{center}

\hypertarget{exercise-5-simulating-multiple-regression}{%
\subsection{Exercise 5 (Simulating Multiple
Regression)}\label{exercise-5-simulating-multiple-regression}}

For this exercise we will simulate data from the following model:

\[
Y_i = \beta_0 + \beta_1 x_{i1} + \beta_2 x_{i2} + \beta_3 x_{i3} + \beta_4 x_{i4} + \beta_5 x_{i5} + \epsilon_i
\]

Where \(\epsilon_i \sim N(0, \sigma^2).\) Also, the parameters are known
to be:

\begin{itemize}
\tightlist
\item
  \(\beta_0 = 2\)
\item
  \(\beta_1 = -0.75\)
\item
  \(\beta_2 = 1.5\)
\item
  \(\beta_3 = 0\)
\item
  \(\beta_4 = 0\)
\item
  \(\beta_5 = 2\)
\item
  \(\sigma^2 = 25\)
\end{itemize}

We will use samples of size \texttt{n\ =\ 42}.

We will verify the distribution of \(\hat{\beta}_2\) as well as
investigate some hypothesis tests.

\textbf{(a)} We will first generate the \(X\) matrix and data frame that
will be used throughout the exercise. Create the following nine
variables:

\begin{itemize}
\tightlist
\item
  \texttt{x0}: a vector of length \texttt{n} that contains all
  \texttt{1}
\item
  \texttt{x1}: a vector of length \texttt{n} that is randomly drawn from
  a normal distribution with a mean of \texttt{0} and a standard
  deviation of \texttt{2}
\item
  \texttt{x2}: a vector of length \texttt{n} that is randomly drawn from
  a uniform distribution between \texttt{0} and \texttt{4}
\item
  \texttt{x3}: a vector of length \texttt{n} that is randomly drawn from
  a normal distribution with a mean of \texttt{0} and a standard
  deviation of \texttt{1}
\item
  \texttt{x4}: a vector of length \texttt{n} that is randomly drawn from
  a uniform distribution between \texttt{-2} and \texttt{2}
\item
  \texttt{x5}: a vector of length \texttt{n} that is randomly drawn from
  a normal distribution with a mean of \texttt{0} and a standard
  deviation of \texttt{2}
\item
  \texttt{X}: a matrix that contains \texttt{x0}, \texttt{x1},
  \texttt{x2}, \texttt{x3}, \texttt{x4}, and \texttt{x5} as its columns
\item
  \texttt{C}: the \(C\) matrix that is defined as \((X^\top X)^{-1}\)
\item
  \texttt{y}: a vector of length \texttt{n} that contains all \texttt{0}
\item
  \texttt{sim\_data}: a data frame that stores \texttt{y} and the
  \textbf{five} \emph{predictor} variables. \texttt{y} is currently a
  placeholder that we will update during the simulation.
\end{itemize}

Report the sum of the diagonal of \texttt{C} as well as the 5th row of
\texttt{sim\_data}. For this exercise we will use the seed \texttt{420}.
Generate the above variables in the order listed after running the code
below to set a seed.

\begin{Shaded}
\begin{Highlighting}[]
\KeywordTok{set.seed}\NormalTok{(}\DecValTok{420}\NormalTok{)}
\NormalTok{sample_size =}\StringTok{ }\DecValTok{42}
\NormalTok{x0 =}\StringTok{ }\KeywordTok{rep}\NormalTok{(}\DecValTok{1}\NormalTok{, sample_size)}
\NormalTok{x1 =}\StringTok{ }\KeywordTok{rnorm}\NormalTok{(}\DataTypeTok{n =}\NormalTok{ sample_size, }\DataTypeTok{mean =} \DecValTok{0}\NormalTok{, }\DataTypeTok{sd =} \DecValTok{2}\NormalTok{)}
\NormalTok{x2 =}\StringTok{ }\KeywordTok{runif}\NormalTok{(}\DataTypeTok{n =}\NormalTok{ sample_size, }\DataTypeTok{min =} \DecValTok{0}\NormalTok{, }\DataTypeTok{max =} \DecValTok{4}\NormalTok{)}
\NormalTok{x3 =}\StringTok{ }\KeywordTok{rnorm}\NormalTok{(}\DataTypeTok{n =}\NormalTok{ sample_size, }\DataTypeTok{mean =} \DecValTok{0}\NormalTok{, }\DataTypeTok{sd =} \DecValTok{1}\NormalTok{)}
\NormalTok{x4 =}\StringTok{ }\KeywordTok{runif}\NormalTok{(}\DataTypeTok{n =}\NormalTok{ sample_size, }\DataTypeTok{min =} \DecValTok{-2}\NormalTok{, }\DataTypeTok{max =} \DecValTok{2}\NormalTok{)}
\NormalTok{x5 =}\StringTok{ }\KeywordTok{rnorm}\NormalTok{(}\DataTypeTok{n =}\NormalTok{ sample_size, }\DataTypeTok{mean =} \DecValTok{0}\NormalTok{, }\DataTypeTok{sd =} \DecValTok{2}\NormalTok{)}
\NormalTok{X =}\StringTok{ }\KeywordTok{cbind}\NormalTok{(x0, x1, x2, x3, x4, x5)}
\NormalTok{C =}\StringTok{ }\KeywordTok{solve}\NormalTok{(}\KeywordTok{t}\NormalTok{(X) }\OperatorTok\StringTok{ }\NormalTok{X)}
\NormalTok{y =}\StringTok{ }\KeywordTok{rep}\NormalTok{(}\DecValTok{0}\NormalTok{, sample_size)}
\NormalTok{sim_data =}\StringTok{ }\KeywordTok{data.frame}\NormalTok{(y, x1, x2, x3, x4, x5)}
\KeywordTok{sum}\NormalTok{(}\KeywordTok{diag}\NormalTok{(C))}
\end{Highlighting}
\end{Shaded}

\begin{verbatim}
## [1] 0.1792443
\end{verbatim}

\begin{Shaded}
\begin{Highlighting}[]
\NormalTok{sim_data[}\DecValTok{5}\NormalTok{, ]}
\end{Highlighting}
\end{Shaded}

\begin{verbatim}
##   y        x1        x2        x3        x4         x5
## 5 0 0.7959582 0.4283331 0.6079313 0.4994189 -0.9018014
\end{verbatim}

\textbf{(b)} Create three vectors of length \texttt{2500} that will
store results from the simulation in part \textbf{(c)}. Call them
\texttt{beta\_hat\_1}, \texttt{beta\_3\_pval}, and
\texttt{beta\_5\_pval}.

\begin{Shaded}
\begin{Highlighting}[]
\NormalTok{num_sims =}\StringTok{ }\DecValTok{2500}
\NormalTok{beta_hat_}\DecValTok{1}\NormalTok{ =}\StringTok{ }\KeywordTok{rep}\NormalTok{(}\DecValTok{0}\NormalTok{, num_sims)}
\NormalTok{beta_}\DecValTok{3}\NormalTok{_pval =}\StringTok{ }\KeywordTok{rep}\NormalTok{(}\DecValTok{0}\NormalTok{, num_sims)}
\NormalTok{beta_}\DecValTok{5}\NormalTok{_pval =}\StringTok{ }\KeywordTok{rep}\NormalTok{(}\DecValTok{0}\NormalTok{, num_sims)}
\end{Highlighting}
\end{Shaded}

\textbf{(c)} Simulate 2500 samples of size \texttt{n\ =\ 42} from the
model above. Each time update the \texttt{y} value of
\texttt{sim\_data}. Then use \texttt{lm()} to fit a multiple regression
model. Each time store:

\begin{itemize}
\tightlist
\item
  The value of \(\hat{\beta}_1\) in \texttt{beta\_hat\_1}
\item
  The p-value for the two-sided test of \(\beta_3 = 0\) in
  \texttt{beta\_3\_pval}
\item
  The p-value for the two-sided test of \(\beta_5 = 0\) in
  \texttt{beta\_5\_pval}
\end{itemize}

\begin{Shaded}
\begin{Highlighting}[]
\NormalTok{beta_}\DecValTok{0}\NormalTok{ =}\StringTok{ }\DecValTok{2}
\NormalTok{beta_}\DecValTok{1}\NormalTok{ =}\StringTok{ }\FloatTok{-0.75}
\NormalTok{beta_}\DecValTok{2}\NormalTok{ =}\StringTok{ }\FloatTok{1.5}
\NormalTok{beta_}\DecValTok{3}\NormalTok{ =}\StringTok{ }\DecValTok{0}
\NormalTok{beta_}\DecValTok{4}\NormalTok{ =}\StringTok{ }\DecValTok{0}
\NormalTok{beta_}\DecValTok{5}\NormalTok{ =}\StringTok{ }\DecValTok{2}
\NormalTok{sigma =}\StringTok{ }\DecValTok{5}

\ControlFlowTok{for}\NormalTok{ (i }\ControlFlowTok{in} \DecValTok{1}\OperatorTok{:}\StringTok{ }\NormalTok{num_sims) \{}
\NormalTok{  sim_data}\OperatorTok{$}\NormalTok{y =}\StringTok{ }\NormalTok{beta_}\DecValTok{0} \OperatorTok{*}\NormalTok{x0 }\OperatorTok{+}\StringTok{ }\NormalTok{beta_}\DecValTok{1} \OperatorTok{*}\StringTok{ }\NormalTok{x1 }\OperatorTok{+}\StringTok{ }\NormalTok{beta_}\DecValTok{2} \OperatorTok{*}\StringTok{ }\NormalTok{x2 }\OperatorTok{+}\StringTok{ }\NormalTok{beta_}\DecValTok{3} \OperatorTok{*}\StringTok{ }\NormalTok{x3 }\OperatorTok{+}\StringTok{ }\NormalTok{beta_}\DecValTok{4} \OperatorTok{*}\StringTok{ }\NormalTok{x4 }\OperatorTok{+}\StringTok{ }\NormalTok{beta_}\DecValTok{5} \OperatorTok{*}\StringTok{ }\NormalTok{x5 }\OperatorTok{+}\StringTok{ }
\StringTok{      }\KeywordTok{rnorm}\NormalTok{(}\DataTypeTok{n =}\NormalTok{ sample_size, }\DataTypeTok{mean =} \DecValTok{0}\NormalTok{, }\DataTypeTok{sd =}\NormalTok{ sigma)}
\NormalTok{  sim_model =}\StringTok{ }\KeywordTok{lm}\NormalTok{(y }\OperatorTok{~}\StringTok{ }\NormalTok{., }\DataTypeTok{data =}\NormalTok{ sim_data)}

\NormalTok{  beta_hat_}\DecValTok{1}\NormalTok{[i] =}\StringTok{ }\KeywordTok{coef}\NormalTok{(sim_model)[}\DecValTok{2}\NormalTok{]}
\NormalTok{  beta_}\DecValTok{3}\NormalTok{_pval[i] =}\StringTok{ }\KeywordTok{summary}\NormalTok{(sim_model)}\OperatorTok{$}\NormalTok{coefficients[}\StringTok{"x3"}\NormalTok{, }\StringTok{"Pr(>|t|)"}\NormalTok{]}
\NormalTok{  beta_}\DecValTok{5}\NormalTok{_pval[i] =}\StringTok{ }\KeywordTok{summary}\NormalTok{(sim_model)}\OperatorTok{$}\NormalTok{coefficients[}\StringTok{"x5"}\NormalTok{, }\StringTok{"Pr(>|t|)"}\NormalTok{]}
\NormalTok{\}}
\end{Highlighting}
\end{Shaded}

\textbf{(d)} Based on the known values of \(X\), what is the true
distribution of \(\hat{\beta}_1\)? The true distribution of
\(\hat{\beta}_1\) is

\begin{Shaded}
\begin{Highlighting}[]
\NormalTok{(}\DataTypeTok{covariance =}\NormalTok{ sigma}\OperatorTok{^}\DecValTok{2} \OperatorTok{*}\StringTok{ }\NormalTok{C[}\DecValTok{1}\OperatorTok{+}\DecValTok{1}\NormalTok{, }\DecValTok{1}\OperatorTok{+}\DecValTok{1}\NormalTok{])}
\end{Highlighting}
\end{Shaded}

\begin{verbatim}
## [1] 0.1892102
\end{verbatim}

\begin{Shaded}
\begin{Highlighting}[]
\NormalTok{(}\DataTypeTok{mean =}\NormalTok{ beta_}\DecValTok{1}\NormalTok{)}
\end{Highlighting}
\end{Shaded}

\begin{verbatim}
## [1] -0.75
\end{verbatim}

\textbf{(e)} Calculate the mean and variance of \texttt{beta\_hat\_1}.
Are they close to what we would expect? Plot a histogram of
\texttt{beta\_hat\_1}. Add a curve for the true distribution of
\(\hat{\beta}_1\). Does the curve seem to match the histogram?

\begin{Shaded}
\begin{Highlighting}[]
\KeywordTok{mean}\NormalTok{(beta_hat_}\DecValTok{1}\NormalTok{)}
\end{Highlighting}
\end{Shaded}

\begin{verbatim}
## [1] -0.7461209
\end{verbatim}

\begin{Shaded}
\begin{Highlighting}[]
\KeywordTok{var}\NormalTok{(beta_hat_}\DecValTok{1}\NormalTok{)}
\end{Highlighting}
\end{Shaded}

\begin{verbatim}
## [1] 0.1853515
\end{verbatim}

\begin{Shaded}
\begin{Highlighting}[]
\KeywordTok{hist}\NormalTok{(beta_hat_}\DecValTok{1}\NormalTok{, }\DataTypeTok{prob =} \OtherTok{TRUE}\NormalTok{, }\DataTypeTok{breaks =} \DecValTok{20}\NormalTok{, }\DataTypeTok{xlab =} \KeywordTok{expression}\NormalTok{(}\KeywordTok{hat}\NormalTok{(beta)[}\DecValTok{1}\NormalTok{]), }\DataTypeTok{main =} \StringTok{""}\NormalTok{, }\DataTypeTok{border =} \StringTok{"orange"}\NormalTok{)}
\KeywordTok{curve}\NormalTok{(}\KeywordTok{dnorm}\NormalTok{(x, }\DataTypeTok{mean =}\NormalTok{ beta_}\DecValTok{1}\NormalTok{, }\DataTypeTok{sd =} \KeywordTok{sqrt}\NormalTok{(sigma }\OperatorTok{^}\DecValTok{2} \OperatorTok{*}\StringTok{ }\NormalTok{C[}\DecValTok{1}\OperatorTok{+}\DecValTok{1}\NormalTok{, }\DecValTok{1}\OperatorTok{+}\DecValTok{1}\NormalTok{])), }\DataTypeTok{col =} \StringTok{"green"}\NormalTok{, }\DataTypeTok{add =} \OtherTok{TRUE}\NormalTok{, }\DataTypeTok{lwd =} \DecValTok{2}\NormalTok{)}
\end{Highlighting}
\end{Shaded}

\includegraphics{w04-hw-bollamr2_files/figure-latex/unnamed-chunk-27-1.pdf}
The results are close to our expectations.

The curve seems to match the histogram. \textbf{(f)} What proportion of
the p-values stored in \texttt{beta\_3\_pval} is less than 0.10? Is this
what you would expect?

\begin{Shaded}
\begin{Highlighting}[]
\KeywordTok{mean}\NormalTok{(beta_}\DecValTok{3}\NormalTok{_pval }\OperatorTok{<}\StringTok{ }\FloatTok{0.10}\NormalTok{)}
\end{Highlighting}
\end{Shaded}

\begin{verbatim}
## [1] 0.096
\end{verbatim}

Yes. This matches our expectaions, since \(\beta_3\) \(\neq\) 0 meaning
that 9.6\% of the p-values are significant at alpha = 0.10

\textbf{(g)} What proportion of the p-values stored in
\texttt{beta\_5\_pval} is less than 0.01? Is this what you would expect?

\begin{Shaded}
\begin{Highlighting}[]
\KeywordTok{mean}\NormalTok{(beta_}\DecValTok{5}\NormalTok{_pval }\OperatorTok{<}\StringTok{ }\FloatTok{0.01}\NormalTok{)}
\end{Highlighting}
\end{Shaded}

\begin{verbatim}
## [1] 0.7956
\end{verbatim}

Yes. This matches our expectations, since \(\beta_3\) \(\neq\) 0.

\end{document}
